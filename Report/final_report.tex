% !TeX program = xelatex
% !TEX TS-program = xelatex
% Compile with: xelatex final_report.tex (twice for TOC)
\documentclass[11pt,a4paper]{ctexart}
\usepackage{iftex}
\ifPDFTeX
  \PackageError{final_report}{請以 XeLaTeX 編譯本文件}{請改用 xelatex 指令或 latexmk 設定為 xelatex}
\fi

% Page and typography
\usepackage[a4paper,margin=1in]{geometry}
\usepackage{setspace}
\setstretch{1.2}
% Relax line-breaking a bit to avoid minor overfull boxes
\emergencystretch=2pt
\usepackage{hyperref}
\hypersetup{colorlinks=true, linkcolor=black, urlcolor=black, citecolor=black}
\usepackage{graphicx}
% Ensure images are included (not in draft) and set search paths
\setkeys{Gin}{draft=false}
\graphicspath{{./}{/Users/jesse/Documents/School Work/高等資料庫/cloud-infrastructure-analysis/Report/}}
\usepackage{booktabs}
\usepackage{array}
\usepackage{enumitem}
\usepackage{float}
\usepackage{subcaption}
\usepackage{tocloft}
% Tighten TOC spacing so last entry doesn't spill to next page
\setlength{\cftbeforesecskip}{2pt}
\setlength{\cftbeforesubsecskip}{1pt}
\setlength{\cftaftertoctitleskip}{8pt}

% Listings for Cypher and code
\usepackage{listings}
\usepackage{xcolor}
\definecolor{codebg}{RGB}{255,255,255}
\definecolor{codeframe}{RGB}{220,220,220}
\lstdefinelanguage{Cypher}{
  morekeywords={MATCH,RETURN,WHERE,AND,OR,NOT,WITH,CREATE,MERGE,DETACH,DELETE,SET,ON,EXISTS,UNWIND,OPTIONAL,AS,DISTINCT,TRUE,FALSE,NULL},
  sensitive=true,
  morestring=[b]",
  morecomment=[l]{//}
}
\lstset{
  language=Cypher,
  backgroundcolor=\color{codebg},
  basicstyle=\ttfamily\small,
  keywordstyle=\bfseries\color{black},
  commentstyle=\color{black},
  stringstyle=\color{black},
  frame=single,
  rulecolor=\color{black},
  frameround=tttt,
  breaklines=true,
  columns=fullflexible,
  tabsize=2,
  showstringspaces=false,
  keepspaces=true,
  numbers=left,
  numberstyle=\scriptsize\color{gray},
  xleftmargin=1em,
  framexleftmargin=1em
}

% Title info
\title{雲端基礎設施視覺化分析平台\\圖形資料庫最終報告}
\author{課程:高等資料庫系統\\姓名:梁祐嘉\\學號:01157145\\班級:資工 4B}
\date{2025/10/22}

% Traditional Chinese names for lists and ToC
\ctexset{
  contentsname = 目錄,
  listfigurename = 圖目錄,
  listtablename = 表目錄
}
% Use English-style short label for figure captions
\renewcommand{\figurename}{fig}
% System font setup for better CJK and code coverage (macOS)
% Requires XeLaTeX; ctex loads fontspec under XeLaTeX
\setmainfont{Times New Roman}
\setsansfont{Helvetica}
\setmonofont{Menlo}
\setCJKmainfont{PingFang TC}
\setCJKsansfont{PingFang TC}
\setCJKmonofont{PingFang TC}

\begin{document}
\maketitle

% Make abstract title larger, keep body at normalsize
\renewcommand{\abstractname}{\large 摘要}
\begin{abstract}
\normalsize
本報告提出一套以 Neo4j 為核心的雲端基礎設施知識圖譜平台,旨在以圖形資料模型整合與分析專案在 AWS 等雲端環境中的複雜資源、關聯與依賴,並提供資安漏洞分析、故障衝擊分析與成本優化核心情境之查詢與分析。系統採用 Python 腳本透過雲端 API 擷取設定資料,轉換為節點與關係後匯入 Neo4j,以圖為中心進行視覺化與查詢。本專案成功實現三大功能完整檢測,系統無警告運行,提供詳細的分析報告和具體的問題檢測建議。
\end{abstract}

\tableofcontents
\clearpage

\section{簡介}
現代雲端環境涵蓋數百至數千個互相連結的資源(virtual hosts, databases, firewalls, load balancers, etc),其依賴、權限與網路關係錯綜複雜。傳統清單式或單點儀表板難以完整呈現多跳關聯,導致在資安風險評估、故障衝擊分析與成本優化等任務上面臨挑戰。本專案以雲端基礎設施知識圖譜為核心,透過 Neo4j 的圖形處理能力,將基礎設施模型化並提供直觀、可深度查詢的分析介面。

\section{Neo4j 產品與服務}
\subsection{使用的 Neo4j 產品}
本專案使用 \textbf{Neo4j Community Edition} 作為圖形資料庫核心,具體包括:

\begin{itemize}[leftmargin=1.5em]
    \item \textbf{Neo4j Database}: 開源圖形資料庫引擎
    \item \textbf{Cypher Query Language}: 圖形查詢語言
    \item \textbf{Neo4j Python Driver}: Python 連接驅動程式
    \item \textbf{Neo4j Browser}: 圖形資料庫管理介面
\end{itemize}

\subsection{技術架構}
整體資料流程如下:
\begin{enumerate}[leftmargin=1.5em]
    \item \textbf{資料擷取}:以 Python 透過雲端 API/CLI(如 AWS Boto3)定期匯出雲端帳號內資源設定(JSON)。
    \item \textbf{資料轉換與載入}:解析 JSON,將資源轉為節點(Node)與關係(Relationship),並以 Neo4j Driver 匯入。
    \item \textbf{查詢與分析}:於 Neo4j Browser 以 Cypher 進行分析與探索。
\end{enumerate}

\section{原始資料格式與來源}
\subsection{資料來源}
\begin{enumerate}[leftmargin=1.5em]
    \item \textbf{AWS API}: 透過 Boto3 SDK 擷取真實 AWS 資源
    \item \textbf{Mock 資料}: 模擬 AWS 環境的測試資料
    \item \textbf{增強模擬資料}: 包含完整測試場景的模擬資料
\end{enumerate}

\subsection{原始資料格式}
原始資料採用 JSON 格式,包含以下主要結構:

\begin{verbatim}
{
  "ec2_instances": {
    "Instances": [
      {
        "InstanceId": "i-1234567890abcdef0",
        "InstanceType": "t3.micro",
        "State": {"Name": "running"},
        "PublicIpAddress": "54.123.45.67",
        "SecurityGroups": [...],
        "Tags": [...]
      }
    ]
  },
  "security_groups": {
    "SecurityGroups": [...]
  },
  "vpcs": {
    "Vpcs": [...]
  },
  "subnets": {
    "Subnets": [...]
  },
  "ebs_volumes": {
    "Volumes": [...]
  },
  "s3_buckets": {
    "Buckets": [...]
  }
}
\end{verbatim}

\section{圖形資料模型設計}
\subsection{核心節點(Nodes)}
\begin{itemize}[leftmargin=1.5em]
    \item \texttt{:EC2Instance}:屬性包含 \texttt{InstanceID}, \texttt{Name}, \texttt{State}, \texttt{PublicIP}, \texttt{InstanceType}。
    \item \texttt{:SecurityGroup}:屬性包含 \texttt{GroupID}, \texttt{GroupName}, \texttt{Description}, \texttt{VpcId}。
    \item \texttt{:SecurityRule}:屬性包含 \texttt{RuleID}, \texttt{Protocol}, \texttt{PortRange}, \\
          \texttt{SourceCIDR}, \texttt{Direction}。
    \item \texttt{:VPC}:屬性包含 \texttt{VpcId}, \texttt{Name}, \texttt{CidrBlock}, \texttt{State}。
    \item \texttt{:Subnet}:屬性包含 \texttt{SubnetId}, \texttt{VpcId}, \texttt{AvailabilityZone}, \texttt{CidrBlock}。
    \item \texttt{:EBSVolume}:屬性包含 \texttt{VolumeId}, \texttt{Size}, \texttt{VolumeType}, \texttt{State}, \texttt{Encrypted}。
    \item \texttt{:S3Bucket}:屬性包含 \texttt{Name}, \texttt{CreationDate}, \texttt{Arn}。
\end{itemize}

\subsection{核心關係(Relationships)}
\begin{itemize}[leftmargin=1.5em]
    \item \texttt{(EC2Instance)-[:IS\_MEMBER\_OF]->(SecurityGroup)}
    \item \texttt{(SecurityGroup)-[:HAS\_RULE]->(SecurityRule)}
    \item \texttt{(EC2Instance)-[:LOCATED\_IN]->(Subnet)},\\
          \texttt{(Subnet)-[:LOCATED\_IN]->(VPC)}
    \item \texttt{(EBSVolume)-[:ATTACHES\_TO]->(EC2Instance)}
\end{itemize}

\section{核心分析功能與範例查詢}
本系統聚焦三大分析場景:資安漏洞分析、故障衝擊分析與成本優化分析。以下提供代表性 Cypher 查詢。

\subsection{資安漏洞分析(Security Vulnerability Analysis)}
\textbf{目標}:找出所有暴露於公網且開啟高風險連接埠(如 SSH:22, RDP:3389)的主機。

\begin{lstlisting}[language=Cypher,caption={尋找允許 0.0.0.0/0 存取 22 埠之主機}]
// 找出所有允許從任何 IP (0.0.0.0/0) 存取 22 號連接埠的主機
MATCH (instance:EC2Instance)-[:IS_MEMBER_OF]->(sg:SecurityGroup),
      (sg)-[:HAS_RULE]->(rule:SecurityRule)
WHERE rule.sourcecidr = '0.0.0.0/0' AND rule.portrange CONTAINS '22'
RETURN instance.name, instance.instanceid, instance.publicip
\end{lstlisting}

\subsection{故障衝擊分析(Failure Impact Analysis)}
\textbf{目標}:識別關鍵節點和網路拓撲結構。

\begin{lstlisting}[language=Cypher,caption={關鍵節點識別}]
// 找出連接數最多的節點(關鍵節點)
MATCH (n)
WITH n, COUNT { (n)--() } as connection_count
WHERE connection_count > 2
RETURN labels(n)[0] as node_type, n.instanceid, 
       n.name, n.vpcid, connection_count
ORDER BY connection_count DESC
\end{lstlisting}

\subsection{成本優化分析(Cost Optimization Analysis)}
\textbf{目標}:找出帳號中的「孤兒硬碟」(Orphaned EBS Volumes)。

\begin{lstlisting}[language=Cypher,caption={找出未連接至任何 EC2 的 EBS 磁碟}]
MATCH (vol:EBSVolume)
WHERE NOT (vol)-[:ATTACHES_TO]->(:EC2Instance)
  AND vol.state = 'available'
RETURN vol.volumeid, vol.size, vol.volumetype, vol.region
ORDER BY vol.size DESC
\end{lstlisting}

\section{系統介面與操作範例}
\subsection{命令行介面}
\begin{verbatim}
# 執行綜合分析(推薦)
python main.py --mode comprehensive-analyze

# 使用模擬資料進行完整分析
python main.py --mode full --mock

# 執行特定分析類型
python main.py --mode advanced-analyze --analysis-type security
python main.py --mode advanced-analyze --analysis-type cost

# 啟動視覺化儀表板
python main.py --mode dashboard --host 0.0.0.0 --port 8050
\end{verbatim}

\subsection{分析結果範例}
系統成功檢測出以下問題:
\begin{itemize}[leftmargin=1.5em]
    \item \textbf{安全分析}:12 個過度寬鬆的安全群組規則,6 個未加密資源,16 個孤兒安全群組,0 個暴露服務(已修正)
    \item \textbf{故障分析}:28 個關鍵節點,66 個單點故障,網路冗餘性分析,171 個節點,111 個關係
    \item \textbf{成本優化}:22 個孤兒 EBS 磁碟(總大小 10,268 GB,預估月成本 \$1,026.8),16 個未使用安全群組,5 個停止實例
\end{itemize}

\section{實作要點}
\subsection{擷取與載入}
\begin{itemize}[leftmargin=1.5em]
    \item \textbf{擷取頻率與版本控管}:定期擷取 JSON 並保留版本,以支援變更比對與回溯。
    \item \textbf{ID 去重與關聯完整性}:以雲端資源原生 ID 作為主鍵,避免重覆匯入;匯入順序先節點後關係。
    \item \textbf{安全性}:妥善保護 API 金鑰,避免將敏感設定納入版本庫。
\end{itemize}

\subsection{查詢效能}
\begin{itemize}[leftmargin=1.5em]
    \item 針對高選擇性屬性(如 \texttt{InstanceID}, \texttt{GroupID})建立索引或唯一性約束。
    \item 對常見路徑查詢調整模式與方向性,減少掃描範圍。
\end{itemize}

\section{結論}
本專案成功實現了一個基於 Neo4j 圖形資料庫的雲端基礎設施分析平台,具備完整的圖形資料模型、三大核心分析功能,系統無警告運行,能夠有效檢測出雲端基礎設施中的安全風險、故障點和成本浪費,為基礎設施管理提供數據驅動的決策支援。

\end{document}
