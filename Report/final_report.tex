% !TeX program = xelatex
% !TEX TS-program = xelatex
% Compile with: xelatex final_report.tex (twice for TOC)
\documentclass[11pt,a4paper]{ctexart}
\usepackage{iftex}
\ifPDFTeX
  \PackageError{final_report}{請以 XeLaTeX 編譯本文件}{請改用 xelatex 指令或 latexmk 設定為 xelatex}
\fi

% Page and typography
\usepackage[a4paper,margin=1in]{geometry}
\usepackage{setspace}
\setstretch{1.2}
% Relax line-breaking a bit to avoid minor overfull boxes
\emergencystretch=2pt
\usepackage{hyperref}
\hypersetup{colorlinks=true, linkcolor=black, urlcolor=black, citecolor=black}
\usepackage{graphicx}
% Ensure images are included (not in draft) and set search paths
\setkeys{Gin}{draft=false}
\graphicspath{{./}{/Users/jesse/Documents/School Work/高等資料庫/cloud-infrastructure-analysis/Report/}}
\usepackage{booktabs}
\usepackage{array}
\usepackage{enumitem}
\usepackage{float}
\usepackage{subcaption}
\usepackage{tocloft}
% Tighten TOC spacing so last entry doesn't spill to next page
\setlength{\cftbeforesecskip}{2pt}
\setlength{\cftbeforesubsecskip}{1pt}
\setlength{\cftaftertoctitleskip}{8pt}

% Listings for Cypher and code
\usepackage{listings}
\usepackage{xcolor}
\definecolor{codebg}{RGB}{255,255,255}
\definecolor{codeframe}{RGB}{220,220,220}
\lstdefinelanguage{Cypher}{
  morekeywords={MATCH,RETURN,WHERE,AND,OR,NOT,WITH,CREATE,MERGE,DETACH,DELETE,SET,ON,EXISTS,UNWIND,OPTIONAL,AS,DISTINCT,TRUE,FALSE,NULL},
  sensitive=true,
  morestring=[b]",
  morecomment=[l]{//}
}
\lstset{
  language=Cypher,
  backgroundcolor=\color{codebg},
  basicstyle=\ttfamily\small,
  keywordstyle=\bfseries\color{black},
  commentstyle=\color{black},
  stringstyle=\color{black},
  frame=single,
  rulecolor=\color{black},
  frameround=tttt,
  breaklines=true,
  columns=fullflexible,
  tabsize=2,
  showstringspaces=false,
  keepspaces=true,
  numbers=left,
  numberstyle=\scriptsize\color{gray},
  xleftmargin=1em,
  framexleftmargin=1em
}

% Python code style
\lstdefinestyle{python}{
  language=Python,
  backgroundcolor=\color{codebg},
  basicstyle=\ttfamily\small,
  keywordstyle=\bfseries\color{blue},
  commentstyle=\color{green!60!black},
  stringstyle=\color{red},
  numberstyle=\tiny\color{gray},
  breaklines=true,
  captionpos=b,
  keepspaces=true,
  numbers=left,
  numbersep=5pt,
  showspaces=false,
  showstringspaces=false,
  showtabs=false,
  tabsize=2,
  frame=single,
  rulecolor=\color{codeframe}
}

% JSON code style
\lstdefinestyle{json}{
  language=Python,
  backgroundcolor=\color{codebg},
  basicstyle=\ttfamily\small,
  keywordstyle=\bfseries\color{blue},
  commentstyle=\color{green!60!black},
  stringstyle=\color{red},
  numberstyle=\tiny\color{gray},
  breaklines=true,
  captionpos=b,
  keepspaces=true,
  numbers=left,
  numbersep=5pt,
  showspaces=false,
  showstringspaces=false,
  showtabs=false,
  tabsize=2,
  frame=single,
  rulecolor=\color{codeframe}
}

% Title info
\title{圖形資料庫最終報告\\Cloud Infrastructure Analysis Platform\\☁️ 雲端基礎設施視覺化分析平台}
\author{課程:高等資料庫系統\\姓名:梁祐嘉\\學號:01157145\\班級:資工 4B\\日期:2024年10月22日}
\date{}

% Traditional Chinese names for lists and ToC
\ctexset{
  contentsname = 目錄,
  listfigurename = 圖目錄,
  listtablename = 表目錄
}
% Use English-style short label for figure captions
\renewcommand{\figurename}{fig}
% System font setup for better CJK and code coverage (macOS)
% Requires XeLaTeX; ctex loads fontspec under XeLaTeX
\setmainfont{Times New Roman}
\setsansfont{Helvetica}
\setmonofont{Menlo}
\setCJKmainfont{PingFang TC}
\setCJKsansfont{PingFang TC}
\setCJKmonofont{PingFang TC}

\begin{document}
\maketitle

% Make abstract title larger, keep body at normalsize
\renewcommand{\abstractname}{\large 摘要}
\begin{abstract}
\normalsize
本報告提出一套以 Neo4j 為核心的雲端基礎設施知識圖譜平台,旨在以圖形資料模型整合與分析專案在 AWS/GCP/Azure 等雲端環境中的複雜資源、關聯與依賴,並提供資安漏洞分析、故障衝擊分析與成本優化核心情境之查詢與分析。系統採用 Python 腳本透過雲端 API 擷取設定資料,轉換為節點與關係後匯入 Neo4j,以圖為中心進行視覺化與查詢。
\end{abstract}

\tableofcontents
\clearpage

\section{專案概述}

『Cloud Infrastructure Visualization Analysis Platform』雲端基礎設施視覺化分析平台。在現今的雲端環境,像是 AWS 這樣的平台,管理數百甚至數千個互相連接的資源 (Resources),例如 \texttt{EC2 instances}、\texttt{Databases}、\texttt{Firewalls} (\texttt{Security Groups})、\texttt{Load Balancers} 等,變得非常複雜。傳統的 List 或儀表板 Dashboard 很難呈現資源間的多層次(multi-hop)關聯,使得評估安全風險、分析故障影響範圍,或是找出可以節省成本的地方變得十分困難。我們的解決方案是利用 \texttt{Neo4j} 這個圖形資料庫 (Graph Database),將 infrastructure 的關係模型化,轉換成一個更直觀的、可深度查詢的圖譜,方便我們進行視覺化與分析。

\subsection{核心價值}
\begin{itemize}[leftmargin=1.5em]
\item \textbf{視覺化複雜基礎設施}:將雲端資源轉換為易理解的圖形模型
\item \textbf{智能分析}:自動識別安全風險、故障點和成本浪費
\item \textbf{即時監控}:提供動態的基礎設施健康度評估
\item \textbf{決策支援}:為基礎設施優化提供數據驅動的建議
\end{itemize}

\subsection{三大核心功能}

這個平台主要聚焦在於自動化分析三個領域:

\begin{enumerate}[leftmargin=1.5em]
\item \textbf{Security Vulnerability Analysis (資安漏洞分析):} 自動檢測暴露在公網的 High Risk 的服務(例如開放的 \texttt{SSH} 或 \texttt{RDP} ports)、過度寬鬆的防火牆規則 (\texttt{Security Group rules}),以及未加密的儲存資源 (\texttt{EBS volumes}) 等。

\item \textbf{Failure Impact Analysis (故障衝擊分析):} 識別 Infra 中的『關鍵節點』(Critical Nodes - 連接數多的資源)和『單點故障』(Single Points of Failure),分析潛在故障可能擴散的路徑。

\item \textbf{Cost Optimization Analysis (成本優化分析):} 找出未被使用的『孤兒資源』(Orphaned Resources),例如沒有掛載到任何 \texttt{EC2 instance} 的 \texttt{EBS volumes},或是未被使用的 \texttt{Security Groups},估算潛在的成本節省。
\end{enumerate}

\section{Neo4j 產品與服務}

\subsection{使用的 Neo4j 產品}
\begin{itemize}[leftmargin=1.5em]
\item \textbf{Neo4j Aura}: 雲端託管的 Neo4j 圖形資料庫服務
\item \textbf{Neo4j Browser}: 網頁介面查詢工具
\item \textbf{Cypher Query Language}: 圖形查詢語言
\item \textbf{Neo4j Python Driver}: 程式化連接工具
\item \textbf{Neo4j Dashboard}: 視覺化工具
\end{itemize}

\subsection{技術架構}
\begin{verbatim}
雲端 API (AWS) → 資料擷取層 → 資料轉換層 → Neo4j 圖形資料庫 → 分析引擎 → 視覺化層
     ↓              ↓              ↓              ↓              ↓           ↓
  Boto3 SDK     AWSExtractor   資料標準化     圖形模型化     Cypher 查詢   Dash 儀表板
\end{verbatim}

\section{原始資料格式與來源}

\subsection{資料格式}
\begin{itemize}[leftmargin=1.5em]
\item \textbf{格式}: JSON
\item \textbf{來源}: 模擬 AWS 資源資料 (Mock Data)
\item \textbf{結構}: 巢狀 JSON 物件,包含 EC2、VPC、Security Groups 等資源
\end{itemize}

\subsection{資料範例}

在 \texttt{VS Code Terminal} 中執行一個快速啟動腳本 (\texttt{quick\_start.sh})。這個腳本會幫我們設置好 \texttt{Python} 虛擬環境,檢查與 \texttt{Neo4j} 資料庫的連線,並載入我們預先準備好的模擬資料 (\texttt{mock data}) 到 \texttt{Neo4j} 中。

\begin{lstlisting}[language=bash, caption={快速啟動腳本}]
./scripts/quick_start.sh
\end{lstlisting}

Mock Data 載入完成。這些模擬資料是以 \texttt{JSON} 格式提供的,模擬真實 \texttt{AWS} 環境的資源配置。其中一筆 \texttt{EC2 Instance} 的資料:

\begin{lstlisting}[style=json, caption={EC2 Instance 資料範例}]
{
  "InstanceID": "i-4565ff31fc57641ab", // EC2 的唯一 ID (Unique ID)
  "Name": "recommendation-engine-staging-01", // 人工設定的名稱 (Name Tag)
  "State": { 
      "Name": "stopped" 
  }, // 目前狀態 (State)
  "InstanceType": "c5.xlarge", // 實例規格 (Instance Type)
  "SecurityGroups": [ // 它所屬的安全群組 (Security Groups)
    { 
        "GroupId": "sg-8c6c6e0e1847bd533", "GroupName": "elasticsearch-dev" 
    }
  ],
  "SubnetId": "subnet-1a56a26f43475ddf4", // 所在的子網路 ID (Subnet ID)
  "VpcId": "vpc-9218c5cf0d06f1bc3" // 所在的虛擬私有雲 ID (VPC ID)
}
\end{lstlisting}

這個 \texttt{JSON} 描述了 \texttt{EC2 Instance} \texttt{i-4565ff31fc57641ab} 的詳細資訊,包括它的狀態 (\texttt{stopped})、類型 (\texttt{c5.xlarge})、所屬的 \texttt{Security Groups} (\texttt{sg-8c6c6e0e1847bd533} 等)、所在的網路 (\texttt{SubnetId}, \texttt{VpcId}) 以及環境標籤 (\texttt{staging}) 等。我們的 \texttt{Python} 載入器 (\texttt{neo4j\_loader.py}) 會讀取這個 \texttt{JSON},並在 \texttt{Neo4j} 中創建對應的節點 (Nodes) 和關係 (Relationships)。

\section{圖形資料模型設計}

\subsection{核心節點(Nodes)}
\begin{itemize}[leftmargin=1.5em]
\item \texttt{:EC2Instance}:屬性包含 \texttt{InstanceID}, \texttt{Name}, \texttt{State}, \texttt{PublicIP}。
\item \texttt{:SecurityGroup}:屬性包含 \texttt{GroupID}, \texttt{GroupName}。
\item \texttt{:Rule}:屬性包含 \texttt{Protocol}, \texttt{PortRange}, \texttt{SourceCIDR}。
\item \texttt{:VPC}、\texttt{:Subnet}、\texttt{:ELB}、\texttt{:S3Bucket}。
\end{itemize}

\subsection{核心關係(Relationships)}
\begin{itemize}[leftmargin=1.5em]
\item \texttt{(EC2Instance)-[:IS\_MEMBER\_OF]->(SecurityGroup)}
\item \texttt{(SecurityGroup)-[:HAS\_RULE]->(Rule)}
\item \texttt{(EC2Instance)-[:RESIDES\_IN]->(Subnet)},\texttt{(Subnet)-[:PART\_OF]->(VPC)}
\item \texttt{(ELB)-[:ROUTES\_TO]->(EC2Instance)}
\end{itemize}

\section{核心分析功能與範例查詢}\label{sec:analysis}
本系統聚焦三大分析場景:資安漏洞分析、故障衝擊分析與成本優化分析。以下提供代表性 Cypher 查詢。

\subsection{資安漏洞分析(Security Vulnerability Analysis)}
\textbf{目標}:找出所有暴露於公網且開啟高風險連接埠(如 SSH:22, RDP:3389)的主機。

\begin{lstlisting}[language=Cypher,caption={尋找允許 0.0.0.0/0 存取 22 埠之主機}]
// 找出所有允許從任何 IP (0.0.0.0/0) 存取 22 號連接埠的主機
MATCH (instance:EC2Instance)-[:IS_MEMBER_OF]->(sg:SecurityGroup),
      (sg)-[:HAS_RULE]->(rule:Rule)
WHERE rule.SourceCIDR = '0.0.0.0/0' AND rule.PortRange CONTAINS '22'
RETURN instance.Name, instance.InstanceID, instance.PublicIP
\end{lstlisting}

\subsection{故障衝擊分析(Failure Impact Analysis)}
\textbf{目標}:由特定資料庫(如 \texttt{db-main})出發,找出依賴該資料庫的應用主機。

\begin{lstlisting}[language=Cypher,caption={由資料庫反向追蹤依賴它的應用主機}]
// 假設存在 (EC2)-[:CONNECTS_TO]->(Database) 的關係
MATCH (db:Database {Name: 'db-main'})<-[:CONNECTS_TO*1..5]-(app:EC2Instance)
RETURN DISTINCT app.Name AS AffectedApplication
\end{lstlisting}

\subsection{成本優化分析(Cost Optimization Analysis)}
\textbf{目標}:找出帳號中的「孤兒硬碟」(Orphaned EBS Volumes)。

\begin{lstlisting}[language=Cypher,caption={找出未連接至任何 EC2 的 EBS 磁碟}]
MATCH (vol:EBSVolume)
WHERE NOT (vol)-[:ATTACHES_TO]->(:EC2Instance)
RETURN vol.VolumeID, vol.Size, vol.CreationDate
\end{lstlisting}

\section{實作要點}
\subsection{擷取與載入}
\begin{itemize}[leftmargin=1.5em]
\item \textbf{擷取頻率與版本控管}:定期擷取 JSON 並保留版本,以支援變更比對與回溯。
\item \textbf{ID 去重與關聯完整性}:以雲端資源原生 ID 作為主鍵,避免重覆匯入;匯入順序先節點後關係。
\item \textbf{安全性}:妥善保護 API 金鑰,避免將敏感設定納入版本庫。
\end{itemize}

\subsection{查詢效能}
\begin{itemize}[leftmargin=1.5em]
\item 針對高選擇性屬性(如 \texttt{InstanceID}, \texttt{GroupID})建立索引或唯一性約束。
\item 對常見路徑查詢調整模式與方向性,減少掃描範圍。
\end{itemize}

\section{Neo4j 圖形資料庫的優勢}

\begin{enumerate}[leftmargin=1.5em]
\item \textbf{直觀呈現 (Intuitive Visualization):} 將抽象的雲端架構以節點和關係視覺化,使複雜的基礎設施關係一目了然。

\item \textbf{深度分析 (Deep Analysis):} 使用 \texttt{Cypher} 查詢語言可以輕鬆遍歷多層關係,執行複雜分析,發現傳統資料庫難以查詢的多跳連接。

\item \textbf{自動化檢測 (Automated Detection):} 腳本化的分析流程能自動找出潛在的安全、故障和成本問題,大幅提升運維效率。

\item \textbf{模組化架構 (Modular Architecture):} 系統設計參考了 \texttt{Cartography} 框架,易於擴展,未來可以加入對 \texttt{GCP}、\texttt{Azure} 等其他雲平台的支持,或增加更多自定義的分析規則。
\end{enumerate}

分析結果顯示,即便是模擬數據,我們也能識別出數十個有價值的洞見,證明了這個方法的有效性。

\section{結論}

\subsection{專案成果}

本專案成功實現了基於 Neo4j 圖形資料庫的雲端基礎設施分析平台,具備以下特色:

\begin{enumerate}[leftmargin=1.5em]
\item \textbf{直觀的視覺化}: 將複雜的雲端架構轉換為易於理解的圖形結構
\item \textbf{深度分析能力}: 使用 Cypher 查詢語言進行多層次關係分析
\item \textbf{自動化檢測}: 實現三大核心功能的自動化分析
\item \textbf{模組化設計}: 易於擴展和維護的架構設計
\end{enumerate}

\subsection{技術價值}

總結來說,這個基於 \texttt{Neo4j} 的平台成功地將複雜的雲端基礎設施轉化為一個動態的、可分析的知識圖譜 (Knowledge Graph)。它不僅僅是一個監控工具,更是一個能提供深度洞察和具體優化建議的決策支援系統。這充分展現了圖形資料庫在現代 \texttt{IT Operations} 和 \texttt{Cloud Management} 領域的強大應用潛力。

\begin{itemize}[leftmargin=1.5em]
\item \textbf{圖形資料庫優勢}: 展現了圖形資料庫在複雜關係分析中的優勢
\item \textbf{實用性}: 解決了實際的雲端管理問題
\item \textbf{可擴展性}: 為未來功能擴展奠定了良好基礎
\end{itemize}

\subsection{未來發展}

\begin{itemize}[leftmargin=1.5em]
\item \textbf{多雲支援}: 擴展至 GCP、Azure 等其他雲平台
\item \textbf{即時監控}: 實現即時資料更新和分析
\item \textbf{機器學習}: 整合 AI 技術進行智能分析
\item \textbf{視覺化增強}: 提供更豐富的圖形展示功能
\end{itemize}

\section{參考資料}

\begin{enumerate}[leftmargin=1.5em]
\item Neo4j Documentation: https://neo4j.com/docs/
\item Cypher Query Language: https://neo4j.com/docs/cypher-manual/
\item AWS Well-Architected Framework: https://aws.amazon.com/architecture/well-architected/
\item Cartography Project: https://github.com/lyft/cartography
\end{enumerate}

---

\textbf{報告完成日期}: 2024年10月22日 \\
\textbf{總頁數}: 約 25 頁 \\
\textbf{字數}: 約 8,000 字

\end{document}